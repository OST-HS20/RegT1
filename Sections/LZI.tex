\section{Nichtlinearität \& Zeitinvariant}
\todo{Woche 11}
\subsection{Linearität}
Linear bedeutet, dass im \textbf{Zeitbereich} ein Faktor $\lambda$ auch eine diesen am Ausgangssignal vorliegt, zudem muss die Gerade durch denn \underline{Nullpunkt}! Im \textbf{Frequenzbereich}, muss Ein/-Ausgangssignal gleiche Frequenz haben!

\subsection{Hysterese}
Um zu entscheiden, ob eine echter Hystereseeffekt (zB dirch Magnetisierung) oder eine Folge der Trägheit des Systems (dynamisch, Tiefpass) ist, kann die Rampe Langsamer oder schneller abgefahren werden. Ist die Hysterese \textit{echte} wird sich die Kennlinie dadurch nicht stark verändern. Im anderen Fall ist die Breite der Hysterese etwa proportional zur Steigung der Rampenfunktion $u(t)$
